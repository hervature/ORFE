\documentclass[12pt]{article}

\usepackage[utf8]{inputenc}
\usepackage{datetime}
\usepackage{amsthm}
\usepackage{amsmath}
\usepackage{amssymb}
\usepackage{enumitem}
\usepackage[english]{babel}
\usepackage{matlab-prettifier}
\usepackage{graphicx}
\usepackage[makeroom]{cancel}
\usepackage{afterpage}
\usepackage{capt-of}
\usepackage{bm}
\usepackage{float}

\DeclareMathOperator*{\argmin}{arg\,min}
\DeclareMathOperator*{\argmax}{arg\,max}

\newcommand\independent{\protect\mathpalette{\protect\independenT}{\perp}}
\def\independenT#1#2{\mathrel{\rlap{$#1#2$}\mkern2mu{#1#2}}}

\newtheoremstyle{colon}{\topsep}{\topsep}{}{}{\bfseries}{:}{ }{}
\theoremstyle{colon}
\newtheorem{exercise}{Exercise}
\newtheorem*{answer}{Answer}

\title{APC 523: Numerical Algorithms for Scientific Computing \\ Homework 2}
\author{Zachary Hervieux-Moore}

\newdate{date}{20}{04}{2019}
\date{\displaydate{date}}

\begin{document}

\maketitle

\clearpage

\begin{exercise}
  \textbf{"Barycentric" interpoloation formula}
\end{exercise}

\begin{answer}
  \begin{enumerate}[label=\alph*)]
    \item Recall the definitions:
      \begin{gather*}
        L_j(x) = \prod_{\substack{k = 0 \\ k \neq j}}^N \frac{x - x_k}{x_j - x_k} \\
        p_N(x) = \sum_{j=0}^N f_j L_j(x)
      \end{gather*}

      Suppose we are interpolating $f(x) = 1$, then our interpolating function $p_N(x) = 1$ and all $f_j = 1$ and thus we have
      \begin{gather*}
        p_N(x) = \sum_{j=0}^N f_j L_j(x) = \sum_{j=0}^N L_j(x) = 1
      \end{gather*}

    \item Substituting their definitions, we get,
      \begin{gather*}
        \frac{w_j^{(N)}}{x-x_j} \cdot L^{(N)}(x) = \frac{\prod_{\substack{k = 0 \\ k \neq j}}^N \frac{1}{x_j - x_k}}{x - x_j} \cdot \prod_{k=0}^N (x - x_k) \\
        = \prod_{\substack{k = 0 \\ k \neq j}} \frac{1}{x_j - x_k} \cdot \prod_{\substack{k = 0 \\ k \neq j}} x - x_k \\
        = \prod_{\substack{k = 0 \\ k \neq j}} \frac{x - x_k}{x_j - x_k} = L_j(x)
      \end{gather*}

      Now, taking the definition, we have
      \begin{gather*}
        L_j(x_j) = \prod_{\substack{k = 0 \\ k \neq j}}^N \frac{x_j - x_k}{x_j - x_k} = 1
      \end{gather*}

      Thus, $L_j(x_j)=1$ and we have $L_k(x_j) = 0$ for $k \neq j$  since $L^{(N)}(x_j)/(x-x_j) = 0$. Thus, the logical branch is
      \begin{gather*}
        L_j(x) = \begin{cases}
          \frac{w_j^{(N)}}{x-x_j} \cdot L^{(N)}(x), \text{ if } x \neq x_j \\
          1, \text{ o.w.}
        \end{cases}
      \end{gather*}

    \item Now, suppose $x = x_k$ for some $k$, then we have
      \begin{gather*}
        p_N(x_k) = \sum_{j=0}^N f_j L_j(x) = \sum_{j=0}^N f_j L_j(x_k) = f_k
      \end{gather*}

      Thus, we get that $p_N(x)$ can be rewritten as when substituing b) as
      \begin{gather*}
        p_N(x) = \begin{cases}
          L^{(N)}(x) \sum_{j=0}^N \frac{w_j^{(N)}}{x-x_j} f_j, \quad x \neq x_0, \mathellipsis, x_N \\
          f_k, \quad x = x_k
        \end{cases}
      \end{gather*}

    \item Using our results
      \begin{gather*}
        \sum_{j=0}^N L_j(x) = 1 \\
        L_j(x) = \frac{w_j^{(N)}}{x-x_j} \cdot L^{(N)}(x)
      \end{gather*}

      then,
      \begin{gather*}
        \sum_{j=0}^N \frac{w_j^{(N)}}{x-x_j} \cdot L^{(N)}(x) = 1 \\
        \implies L^{(N)}(x) \sum_{j=0}^N \frac{w_j^{(N)}}{x-x_j} = 1
      \end{gather*}

      Now, injecting this into the denominator of part c) yields,
      \begin{gather*}
        p_N(x) = \begin{cases}
          \frac{L^{(N)}(x) \sum_{j=0}^N \frac{w_j^{(N)}}{x-x_j} f_j}{L^{(N)}(x) \sum_{j=0}^N \frac{w_j^{(N)}}{x-x_j}}, \quad x \neq x_0, \mathellipsis, x_N \\
          f_k, \quad x = x_k
        \end{cases} \\
        \implies p_N(x) = \begin{cases}
          \frac{\sum_{j=0}^N \frac{w_j^{(N)}}{x-x_j} f_j}{\sum_{j=0}^N \frac{w_j^{(N)}}{x-x_j}}, \quad x \neq x_0, \mathellipsis, x_N \\
          f_k, \quad x = x_k
        \end{cases}
      \end{gather*}

    \item See the implementation in \texttt{question1.ipynb}

    \item See the implementation in \texttt{question1.ipynb}

  \end{enumerate}
\end{answer}

\end{document}