\documentclass[12pt]{article}

\usepackage[utf8]{inputenc}
\usepackage{datetime}
\usepackage{amsthm}
\usepackage{amsmath}
\usepackage{amssymb}
\usepackage{enumitem}
\usepackage[USenglish]{babel}

\newtheoremstyle{colon}{\topsep}{\topsep}{}{}{\bfseries}{:}{ }{}
\theoremstyle{colon}
\newtheorem{exercise}{Exercise}
\newtheorem*{answer}{Answer}

\title{ORFE 526: Probability Theory \\ Chapter 1}
\author{Zachary Hervieux-Moore}

\begin{document}
\maketitle

\clearpage

\begin{exercise}
  Show that an intersection of an arbitrary (countable or uncountable) family of $\sigma$-algebras on $E$ is again a $\sigma$-algebra on $E$. What about unions of $\sigma$-algebras?
\end{exercise}

\begin{answer}
  To be a $\sigma$-algebra, it must be closed under complements and under countable union. Indeed let $\mathcal{E}_n$ be a sequence of $\sigma$-algebras,
  \begin{gather*}
    A \in \bigcap_n \mathcal{E}_n \implies A \in \mathcal{E}_n \ \forall n\\
    \implies A^c \in \mathcal{E}_n \ \forall n \implies A^c \in \bigcap_n \mathcal{E}_n
  \end{gather*}
  Likewise, let $A_m$ be a collection of sets in $\bigcap_n \mathcal{E}_n$,
  \begin{gather*}
    A_m \in \bigcap_n \mathcal{E}_n \ \forall m \implies A_m \in \mathcal{E}_n \ \forall m,n\\
    \implies \bigcup_m A_m \in \mathcal{E}_n \ \forall n \implies \bigcup_m A_m \in \bigcap_n \mathcal{E}_n
  \end{gather*}
  Thus $\sigma$-algebras are closed under intersection. However, they are not closed under union. Consider $\{\emptyset, A, A^c, E\}$ and $\{\emptyset, B, B^c, E\}$ which are both $\sigma$-algebras, but their union $\{\emptyset, A, A^c, B, B^c, E\}$ is not a $\sigma$-algebra since it is missing $A \cup B$.
\end{answer}

\clearpage

\begin{exercise}
  Show that if $\mathcal{E}$ is a $\sigma$-algebra, then
  \begin{enumerate}[label=\roman*)]
    \item $\mathcal{E}$ is a p-system
    \item $\mathcal{E}$ is a d-system if
  \end{enumerate}
\end{exercise}

\begin{answer}
  \leavevmode
  \begin{enumerate}[label=\roman*)]
    \item $\mathcal{E}$ is a p-system if it is non-empty and it is closed under intersection. Obviously, a $\sigma$-algebra is non-empty. It is also closed under intersection due to De Morgan's law that $\bigcap\limits_{n \geq 1} A_i = \Big( \bigcup\limits_{n \geq 1} A_i^c \Big)^c$. As well, $\sigma$-algebras are closed under complementation and union:
      \begin{align*}
          (A_n)_n \subset \mathcal{E} &\implies (A_n^c)_n \subset \mathcal{E} && \text{Since closed under complementation}\\
          &\implies \bigcup\limits_{n \geq 1} A_i^c \in \mathcal{E} && \text{Since closed under union} \\
          &\implies \Big( \bigcup\limits_{n \geq 1} A_i^c \Big)^c \in \mathcal{E} && \text{Since closed under complementation} \\
          &\implies \bigcap\limits_{n \geq 1} A_i \in \mathcal{E}  && \text{By De Morgan's}
      \end{align*}
    \item $\mathcal{E}$ is a d-system if
      \begin{enumerate}[label=\roman*)]
        \item $E \in \mathcal{D}$
        \item if $A, B \in \mathcal{D}$ and $A \subseteq B$, then $B \backslash A \in \mathcal{D}$
        \item if $A_1, A_2, A_3, \mathellipsis \in \mathcal{D}$ and $A_n \subseteq A_{n+1}$, $n \geq 1$, then $A_n \nearrow A \in \mathcal{D}$
      \end{enumerate}

      $E$ is in $\mathcal{E}$ since $A \cup A^c \in \mathcal{E}$ and $A \cup A^c = E$. Now, $B \backslash A = B \cap A^c$ which is in $\mathcal{E}$ since $\sigma$-algebras are closed under complementation and intersection (since they are d-systems by the first part). We also have $A_n \nearrow A \in \mathcal{E}$ since $A_n \nearrow A = \bigcup_n A_n$ and $\sigma$-algebras are closed under union.
  \end{enumerate}
\end{answer}

\clearpage

\begin{exercise}
  Let $\mathcal{D}$ be a a d-system on $E$. Fix $D$ in $\mathcal{D}$ and define
  \begin{center}
    $\widehat{\mathcal{D}} = \{A \in \mathcal{D}: A \cap D \in \mathcal{D}\}$
  \end{center}
  Prove that $\widehat{\mathcal{D}}$ is a d-system.
\end{exercise}

\begin{answer}
  To show that something is a d-system, we must show the following 3 properties:

  \begin{enumerate}[label=\roman*)]
    \item $E \in \mathcal{D}$
    \item if $A, B \in \mathcal{D}$ and $A \subseteq B$, then $ B \backslash A \in \mathcal{D}$
    \item if $A_1, A_2, A_3, \mathellipsis \in \mathcal{D}$ and $A_n \subseteq A_{n+1}$, $n \geq 1$, then $A_n \nearrow A \in \mathcal{D}$
  \end{enumerate}

  Proof of these properties for $\widehat{\mathcal{D}} = \{A \in \mathcal{D}: A \cap D \in \mathcal{D}\}$:

  \begin{enumerate}[label=\roman*)]
    \item Let $A = E \in \mathcal{D}$, then $A \cap D = E \cap D = D$ since the intersection of a set and the universe is just the set. Note that $D \in \mathcal{D}$ by definition. So, $E \cap D \in \mathcal{D}$ which implies $E \in \widehat{\mathcal{D}}$
    \item Let $A, B \in \widehat{\mathcal{D}}$ and $A \subseteq B$, then
      \begin{align*}
        & \text{Since $A, B \in \widehat{\mathcal{D}}$, then } (B \cap D) \in \mathcal{D} \text{ and } (A \cap D) \in \mathcal{D} \\
        & \text{Since  $\mathcal{D}$ is a d-system, then } (B \cap D) \backslash (A \cap D) \in \mathcal{D} \\
        & \text{Note that } (B \cap D) \backslash (A \cap D) =(B \backslash A) \cap D \\
        & \text{So, } (B \backslash A) \cap D \in \mathcal{D} \\
        & \text{Thus, } B \backslash A \in \widehat{\mathcal{D}}
      \end{align*}
    \item Let $A_1, A_2, A_3, \mathellipsis \in \widehat{\mathcal{D}}$ and $A_n \subseteq A_{n+1}$, $n \geq 1$, then
      \begin{align*}
        & \text{Since $A_1, A_2, A_3, \mathellipsis \in \widehat{\mathcal{D}}$, then } A_n \cap D \in \mathcal{D}, n \geq 1 \\
        & \text{Since  $\mathcal{D}$ is a d-system, then } \bigcup\limits_{n \geq 1}(A_n \cap D) \in \mathcal{D} \\
        & \text{Distributing the union we have, } (\cup_{n \geq 1}A_n) \bigcap (\cup_{n \geq 1}D) = A \cap D \\
        & \text{Where } \cup_{n \geq 1}A_n = A \text{ since $\mathcal{D}$ is a d-system} \\
        & \text{So, } A \cap D \in \mathcal{D} \\
        & \text{Thus, } A \in \widehat{\mathcal{D}}
      \end{align*}
  \end{enumerate}

  All 3 properties hold, thus $\widehat{\mathcal{D}}$ is a d-system.
\end{answer}

\clearpage

\begin{exercise}
  Show than an intersection of arbitrary (countable or uncountable) family of d-systems on $E$ is again a d-system on $E$. What about p-systems?
\end{exercise}

\begin{answer}
  To show that something is a d-system, we must show the following 3 properties:

  \begin{enumerate}[label=\roman*)]
    \item $E \in \mathcal{D}$
    \item if $A, B \in \mathcal{D}$ and $A \subseteq B$, then $ B \backslash A \in \mathcal{D}$
    \item if $A_1, A_2, A_3, \mathellipsis \in \mathcal{D}$ and $A_n \subseteq A_{n+1}$, $n \geq 1$, then $A_n \nearrow A \in \mathcal{D}$
  \end{enumerate}

  Let $\mathcal{D}_n$ be a collection of d-systems. Then,
  \begin{enumerate}[label=\roman*)]
    \item Since $\mathcal{D}_n$ are all d-systems, we have
      \begin{gather*}
        E \in \mathcal{D}_n \ \forall n \implies E \in \bigcap_n \mathcal{D}_n
      \end{gather*}
    \item if $A, B \in \mathcal{D}_n$ and $A \subseteq B$, then $B \backslash A \in \mathcal{D}_n \ \forall n$ which implies $B \backslash A \in \bigcap_n \mathcal{D}_n$
    \item if $A_1, A_2, A_3, \mathellipsis \in \mathcal{D}_n$ and $A_m \subseteq A_{m+1}$, $m \geq 1$, then $A_m \nearrow A \in \mathcal{D}_n \ \forall n$ which implies $A_m \nearrow A \in \bigcap_n \mathcal{D}_n$
  \end{enumerate}

  Thus d-systems are closed under intersection. However, p-system need not be closed under intersection. Consider $\{ A, B, A \cap B\}$ and $\{ C, D, C \cap D\}$ which are both p-systems. However, their intersection is $\emptyset$ which is not a p-system.
\end{answer}

\clearpage

\begin{exercise}
  Let $\mathcal{C}$ be a countable partition of $E$. Show that every element of $\sigma \mathcal{C}$ is a countable union of elements taken from $\mathcal{C}$.
\end{exercise}

\begin{answer}
  Since $\mathcal{C}$ is a partition of $E$. Then, for any $A \subset \mathcal{C}$, $A^c$ is the union of every other element of the partition not in $A$. That is, all complements are countable unions of the elements of the partition. Also, there are no intersections since this is a partition. By definition, if $A,B \in \mathcal{C}$ then $A \cap B = \emptyset$. Thus, $\sigma \mathcal{C}$ is simply all the countable unions of the partition $\mathcal{C}$.
\end{answer}

\clearpage

\begin{exercise}
  Let $\mathcal{C}, \mathcal{D} \subset 2^E$. Show the following:
  \begin{enumerate}[label=\roman*)]
    \item $\mathcal{C} \subset \mathcal{D} \implies \sigma\mathcal{C} \subset \sigma\mathcal{D}$
    \item $\mathcal{C} \subset \sigma\mathcal{D} \implies \sigma\mathcal{C} \subset \sigma\mathcal{D}$
    \item If $\mathcal{C} \subset \sigma\mathcal{D}$ and $\mathcal{D} \subset \sigma\mathcal{C} \implies \sigma\mathcal{C} = \sigma\mathcal{D}$
    \item $\mathcal{C} \subset \mathcal{D} \sigma\mathcal{C} \implies \sigma\mathcal{C} = \sigma\mathcal{D}$
  \end{enumerate}
\end{exercise}

\begin{answer}
  \leavevmode
  \begin{enumerate}[label=\roman*)]
    \item By assumption and definition,
      \begin{align*}
        &\mathcal{C} \subset \mathcal{D} \subset \sigma \mathcal{D} \\
        &\text{The definition of $\sigma \mathcal{C}$ is: } \bigcap \mathcal{E} = \sigma \mathcal{C} \text{ where $\mathcal{E}$ is any $\sigma$-algebra containing $\mathcal{C}$} \\
        &\sigma \mathcal{D} \text{ is one such $\mathcal{E}$. Thus, } \sigma \mathcal{C} \subset \sigma \mathcal{D}
      \end{align*}
    \item This follows from the first part. No assumption was made if $\mathcal{D}$ was a $\sigma$-algebra or not.
    \item By the previous part, $\mathcal{C} \subset \sigma\mathcal{D} \implies \sigma\mathcal{C} \subset \sigma\mathcal{D}$ and $\mathcal{D} \subset \sigma\mathcal{C} \implies \sigma\mathcal{D} \subset \sigma\mathcal{C}$. So, $\sigma\mathcal{C} = \sigma\mathcal{D}$.
    \item By the first part, $\mathcal{C} \subset \mathcal{D} \implies \sigma \mathcal{C} \subset \sigma \mathcal{D}$. The second part shows that $\mathcal{D} \subset \sigma\mathcal{C} \implies \sigma \mathcal{D} \subset \sigma \mathcal{C}$. So, $\sigma\mathcal{C} = \sigma\mathcal{D}$.
  \end{enumerate}
\end{answer}

\clearpage

\begin{exercise}
  Show that $\mathcal{B}(\mathbb{R})$ can be generated as
  \begin{enumerate}[label=\roman*)]
    \item $\mathcal{B}(\mathbb{R}) = \sigma\{ (-\infty,x]; x \in \mathbb{R} \}$
    \item $\mathcal{B}(\mathbb{R}) = \sigma\{ (-\infty,x); x \in \mathbb{R} \}$
    \item $\mathcal{B}(\mathbb{R}) = \sigma\{ (x,y]; x,y \in \mathbb{R} \}$
    \item $\mathcal{B}(\mathbb{R}) = \sigma\{ (x,\infty); x \in \mathbb{R} \}$
  \end{enumerate}
\end{exercise}

\begin{answer}
  \leavevmode
  \begin{enumerate}[label=\roman*)]
    \item All open sets are a countable union of open intervals (due to the fact the the rationals are dense but countable in $\mathbb{R}$). Thus, we just have to show that any open interval $(a,b)$ is the countable union (and complement) of intervals of the form $(-\infty, x]$. Consider $a_n = a + \frac{1}{n}$ and $b_n = b - \frac{1}{n}$. Then,
    \begin{gather*}
      (a,b) = \bigcup_{n=1}^\infty (a_n, b_n] = \bigcup_{n=1}^\infty (-\infty, a_n]^c \cap (-\infty, b_n]
    \end{gather*}
    \item We follow the same technique as the first part.
    \begin{gather*}
      (a,b) = \bigcup_{n=1}^\infty [a_n, b_n) = \bigcup_{n=1}^\infty (-\infty, a_n)^c \cap (-\infty, b_n)
    \end{gather*}
    \item We follow the same technique as the first part.
    \begin{gather*}
      (a,b) = \bigcup_{n=1}^\infty (a_n, b_n]
    \end{gather*}
    \item Notice that $(x,\infty)^c = (-\infty,x]$ thus the proof is the complement of the first part.
  \end{enumerate}
\end{answer}

\clearpage

\begin{exercise}
  Let $(E, \mathcal{E})$ be a measurable space. Fix $D$ in $E$ and let
  \begin{gather*}
    D = \mathcal{E} \cap D = \{ A \cap D: A \in \mathcal{E} \}
  \end{gather*}
  Show that $\mathcal{D}$ is a $\sigma$-algebra on $D$.
\end{exercise}

\begin{answer}
  Obviously, $\mathcal{D}$ is non-empty since $D \in \mathcal{D}$. We show closed under complement first. Suppose $A \cap D \in \mathcal{D}$. Then
  \begin{gather*}
    (A \cap D)^c = A^c \cup D^c \\
    \implies (A^c \cup D^c) \cap D = A^c \cap D \bigcup D^c \cap D = A^c \cap D \in \mathcal{D}
  \end{gather*}
  Where the last equality is true because $A^c \in \mathcal{E}$.
  Thus, $(A \cap D)^c \in \mathcal{D}$. Now for countable union. Suppose $A_n \cap D \in \mathcal{D}$. Then,
  \begin{gather*}
    \bigcup_n (A_n \cap D) = \cup_n A_n \bigcap D \in \mathcal{D}
  \end{gather*}
  Since $\cup_n A_n \in \mathcal{E}$.
\end{answer}

\clearpage

\begin{exercise}
  Consider a function $f: E \rightarrow F$. Show the following:
  \begin{enumerate}[label=\roman*)]
    \item $f^{-1}(\emptyset) = \emptyset$ and $f^{-1}(F) = E$
    \item $f^{-1}(B \backslash C) = f^{-1}(B) \backslash f^{-1}(C)$
    \item $f^{-1}(\cup_i B_i) = \cup_i f^{-1}(B_i)$
    \item $f^{-1}(\cap_i B_i) = \cap_i f^{-1}(B_i)$
  \end{enumerate}
  for any $B_i \in 2^E$
\end{exercise}

\begin{answer}
  \leavevmode
  \begin{enumerate}[label=\roman*)]
    \item By definition of a function, it must map an element in $E$ to an element in $F$. Thus, we cannot have $A \neq \emptyset$ and $f(A) = \emptyset$. Thus, $f^{-1}(\emptyset) = \emptyset$. Now, we have by sequence of logic,
      \begin{align*}
        f^{-1}(F) &= \{ A \in E: f(A) \in F \} \\
        &= \{ A \in E: f(A) \notin \emptyset \} \\
        &= \{ A \in E: f(A) \in \emptyset \}^c \\
        &= \{ A \in E: f(A) \in \emptyset \}^c = \emptyset^c = E
      \end{align*}
    \item Using the same technique as the first part,
      \begin{align*}
        f^{-1}(B \backslash C) &= \{ A \in E: f(A) \in B \backslash C \} \\
        &= \{ A \in E: f(A) \in B \cap C^c \} \\
        &= \{ A \in E: f(A) \in B \} \cap \{ A \in E: f(A) \in C^c \} \\
        &= f^{-1}(B) \cap f^{-1}(C^c) = f^{-1}(B) \backslash f^{-1}(C)
      \end{align*}
    \item Using the same technique as the first part,
      \begin{align*}
        f^{-1}(\cup_i B_i) &= \{ A \in E: f(A) \in \cup_i B_i \} \\
        &= \cup_i \{ A \in E: f(A) \in B_i \} \\
        &= \cup_i f^{-1}(B_i)
      \end{align*}
    \item Using the same technique as the first part,
      \begin{align*}
        f^{-1}(\cap_i B_i) &= \{ A \in E: f(A) \in \cap_i B_i \} \\
        &= \cap_i \{ A \in E: f(A) \in B_i \} \\
        &= \cap_i f^{-1}(B_i)
      \end{align*}
  \end{enumerate}
\end{answer}

\clearpage

\begin{exercise}
  Let $(E, \mathcal{E})$ and $(F, \mathcal{F})$ be measurable spaces and consider a function $f: E \rightarrow F$. Define $\mathcal{F}_1 = \{ B \in \mathcal{F}: f^{-1}(B) \in \mathcal{E} \}$. Show that $\mathcal{F}_1$ is a $\sigma$-algebra on $\mathcal{F}$.
\end{exercise}

\begin{answer}
  We prove closed under complements first. Let $A \in \mathcal{F}_1$. Then
  \begin{gather*}
    A \in \mathcal{F}_1 = \{ A \in \mathcal{F}: f^{-1}(A) \in \mathcal{E} \} \\
  \end{gather*}
  But $\mathcal{E}$ is a $\sigma$-algebra, so $f^{-1}(A)^c \in \mathcal{E}$. From the previous exercise
  \begin{gather*}
    f^{-1}(A^c) = f^{-1}(E \backslash A) = f^{-1}(E) \backslash f^{-1}(A) = f^{-1}(A)^c
  \end{gather*}
  So, $f^{-1}(A^c) \in \mathcal{E}$. We also have $A^c \in \mathcal{F}$ since it is also a $\sigma$-algebra. Thus, $A^c \in \mathcal{F}_1$.

  Now to show that it is closed under countable union. Let $A_n \in \mathcal{F}_1$. Then by the previous exercise,
  \begin{gather*}
    f^{-1}(\cup_n A_n) = \cup_n f^{-1}(A_n)
  \end{gather*}
  Since $\mathcal{E}$ is a $\sigma$-algebra, then $\cup_n f^{-1}(A_n) \in \mathcal{E}$. We also have $\cup_n A_n \in \mathcal{F}$ since it is also a $\sigma$-algebra. Thus, $\cup_n A_n \in \mathcal{F}_1$.
\end{answer}

\clearpage

\begin{exercise}
  Let $E$ be a set and $(F, \mathcal{F})$ a measurable space. Consider a function $f: E \rightarrow F$. Define $f^{-1}(\mathcal{F}) = \{ f^{-1}(B) : B \in \mathcal{F} \}$. Prove that:
  \begin{enumerate}[label=(\roman*)]
    \item $f^{-1}(\mathcal{F})$ is a $\sigma$-algebra.
    \item $f^{-1}(\mathcal{F})$ is the smallest $\sigma$-algebra on $E$ such that $f$ is measurable relative to it and $\mathcal{F}$.
  \end{enumerate}
\end{exercise}

\begin{answer}
  \leavevmode
  \begin{enumerate}[label=(\roman*)]
    \item First, note that the inverse image preserves complements and union:
      \par
      Complementation:
      \begin{align*}
        a \in f^{-1}(A^c) &\Longleftrightarrow f(a) \in A^c \\
        &\Longleftrightarrow f(a) \notin A \\
        &\Longleftrightarrow a \notin f^{-1}(A) \\
        &\Longleftrightarrow a \in f^{-1}(A)^c
      \end{align*}
      Union:
      \begin{align*}
        a \in f^{-1}(A \cup B) &\Longleftrightarrow f(a) \in A \cup B \\
        &\Longleftrightarrow f(a) \in A \text{ or } f(a) \in B \\
        &\Longleftrightarrow a \in f^{-1}(A) \text{ or } a \in f^{-1}(B) \\
        &\Longleftrightarrow a \in f^{-1}(A) \cup f^{-1}(B)
      \end{align*}
      We now use these to show $f^{-1}(\mathcal{F})$ is closed under complementation and countable union:
      \par
      Complementation:
      \begin{align*}
        A \in f^{-1}(\mathcal{F}) &\Longleftrightarrow A = f^{-1}(B) \text{ for some } B \\
        &\Longleftrightarrow A^c = f^{-1}(B)^c \\
        &\Longleftrightarrow A^c = f^{-1}(B^c) \text{ , note that } B^c \in \mathcal{F} \text{ since it is $\sigma$-algebra} \\
        &\Longleftrightarrow A^c \in f^{-1}(\mathcal{F})
      \end{align*}
      Union:
      \begin{align*}
        A_n \in f^{-1}(\mathcal{F}) &\Longleftrightarrow A_n = f^{-1}(B_n) \text{ for some } B_n \\
        &\Longleftrightarrow \bigcup_{n \geq 1} A_n = \bigcup_{n \geq 1} f^{-1}(B_n) \\
        &\Longleftrightarrow \bigcup_{n \geq 1} A_n = f^{-1}\big(\bigcup_{n \geq 1} B_n \big) \\
      \end{align*}
      Note that $\bigcup_{n \geq 1} B_n \in \mathcal{F}$ since it is $\sigma$-algebra
      \begin{align*}
        &\Longleftrightarrow \bigcup_{n \geq 1} A_n  \in f^{-1}(\mathcal{F})
      \end{align*}
      Also, $E \in f^{-1}(\mathcal{F})$ since $f^{-1}(F) = E$. Thus, $f^{-1}(\mathcal{F})$ is a $\sigma$-algebra.

      \item Assume there is a smaller $\sigma$-algebra such that $\mathcal{E} \subset f^{-1}(\mathcal{F})$. But we know that $f$ is measurable so, $f^{-1}(\mathcal{F}) \subset \mathcal{E}$. But this contradicts our assumption, therefore there does not exist a smaller $\sigma$-algebra and $f^{-1}(\mathcal{F}) = \mathcal{E}$.
  \end{enumerate}
\end{answer}

\clearpage

\begin{exercise}
  Let $\mathcal{E}$ be a $\sigma$-algebra and consider a sequence $(A_n)_n \subset \mathcal{E}$. Prove that $\bigcap\limits_{n \geq 1} A_n \in \mathcal{E}$.
\end{exercise}

\begin{answer}
  We know by De Morgan's law that $\bigcap\limits_{n \geq 1} A_i = \Big( \bigcup\limits_{n \geq 1} A_i^c \Big)^c$. As well, $\sigma$-algebras are closed under complementation and union:
  \begin{align*}
      (A_n)_n \subset \mathcal{E} &\implies (A_n^c)_n \subset \mathcal{E} && \text{Since closed under complementation}\\
      &\implies \bigcup\limits_{n \geq 1} A_i^c \in \mathcal{E} && \text{Since closed under union} \\
      &\implies \Big( \bigcup\limits_{n \geq 1} A_i^c \Big)^c \in \mathcal{E} && \text{Since closed under complementation} \\
      &\implies \bigcap\limits_{n \geq 1} A_i \in \mathcal{E}  && \text{By De Morgan's}
  \end{align*}
\end{answer}

\clearpage

\begin{exercise}
  Let $f: \mathbb{R} \rightarrow \overline{\mathbb{R}}$ be an increasing function. Show that $f$ is Borel measurable.
\end{exercise}

\begin{answer}
  A function is Borel measurable if for every $r \in \mathbb{R}$, then $E = \{ x: f(x) \leq r\}$ is measurable.
  \begin{align*}
    &\text{Define } b = \sup f^{-1}\big((-\infty, r]\big) \\
    &\text{If } b = \infty \text{, then } E = \mathbb{R} \\
    &\text{If } b = -\infty \text{, then } E = \emptyset \\
    &\text{If } b \in \mathbb{R} \text{, then } E = (-\infty, b] \text{ or } E = (-\infty, b) \\
    &\text{ because } f(x) \leq r \text{ for all } x \in E \text{ since f is increasing.}
  \end{align*}
  All of these $E$'s are elements of the Borel set and hence Borel measurable.
\end{answer}

\clearpage

\begin{exercise}
  Let $(E, \mathcal{E})$ be a measurable space and $f: E \rightarrow \mathbb{R}$ a Borel measurable function.
  \begin{enumerate}[label=(\roman*)]
    \item Show that $|f|$ is measurable
    \item Let $f^+ = \max \{f, 0\}$ and $f^- = -\min \{f, 0\}$. Show that $|f| = f^+ + f^-$
    \item Use (i) and (ii) to show that $f^+$ and $f^-$ are measurable
  \end{enumerate}
\end{exercise}

\begin{answer}
  \leavevmode
  \begin{enumerate}[label=(\roman*)]
    \item We will show that the absolute value does not change measurability
      \begin{align*}
        &\text{For $r < 0$, } \{ x: |f(x)| \leq r\} = \emptyset \\
        &\text{For $r = \infty$, }\{ x: |f(x)| \leq \infty\} = E \\
        &\text{For $r \geq 0$, } \{ x: |f(x)| \leq r\} = \{ x: -r \leq f(x) \leq r\} \\
        &= \{ x: f(x) \leq r\} \bigcap \{ x: f(x) \leq -r\}^c \\
        &\text{Note that the final line is the intersection of two measurable sets} \\
        &\text{ since $f$ itself is measurable. Note that an intersection of two } \\
        &\text{ measurable set is measurable since $\sigma$-algebras are closed under } \\
        &\text{ countable intersection.}
      \end{align*}
      Thus, since for every $r \in \mathbb{R}$, $\{ x: |f(x)| \leq r\}$ is measurable, then $|f|$ is measurable.
    \item Break $f$ into two cases, $f \geq 0$ and $f < 0$:
      \begin{align*}
        &f \geq 0: \text{ Then } \max\{f, 0\} = f \text{ and } -\min\{f,0\} = 0 \\
        &f^+ + f^- = f + 0 = f \\
        &f < 0: \text{ Then } \max\{f, 0\} = 0 \text{ and } -\min\{f,0\} = -f \\
        &f^+ + f^- = 0 - f = -f
      \end{align*}
      Thus, we have $f^+ + f^-$ defined piecewise as:
      $
      f = \begin{cases}
          f & f \geq 0 \\
          -f & f < 0
      \end{cases}
      $
      \par
      Which is the exact definition of $|f|$. Hence, $|f| = f^+ + f^-$
    \item One cleverly notes that we can rewrite $f^+$ and $f^-$ as follows:
      \begin{align*}
        &f^+ = \frac{|f| + f}{2} \\
        &f^- = \frac{|f| - f}{2}
      \end{align*}
      And note from lecture that the sum of two measurable functions is measurable.
  \end{enumerate}
\end{answer}

\end{document}