\documentclass[12pt]{article}

\usepackage[utf8]{inputenc}
\usepackage{datetime}
\usepackage{amsthm}
\usepackage{amsmath}
\usepackage{amssymb}
\usepackage{enumitem}
\usepackage[english]{babel}
\usepackage{matlab-prettifier}
\usepackage{graphicx}
\usepackage[makeroom]{cancel}
\usepackage{afterpage}
\usepackage{capt-of}
\usepackage{bm}
\usepackage{float}

\DeclareMathOperator*{\argmin}{arg\,min}
\DeclareMathOperator*{\argmax}{arg\,max}

\newcommand\independent{\protect\mathpalette{\protect\independenT}{\perp}}
\def\independenT#1#2{\mathrel{\rlap{$#1#2$}\mkern2mu{#1#2}}}

\newtheoremstyle{colon}{\topsep}{\topsep}{}{}{\bfseries}{:}{ }{}
\theoremstyle{colon}
\newtheorem{exercise}{Exercise}
\newtheorem*{answer}{Answer}

\title{ELE 535: Machine Learning and Pattern Recognition \\ Homework 2}
\author{Zachary Hervieux-Moore}

\newdate{date}{01}{10}{2018}
\date{\displaydate{date}}

\begin{document}

\maketitle

\clearpage

\begin{exercise}
  Let $u \in \mathbb{R}^m$, $v \in \mathbb{R}^n$, and $A \in \mathbb{R}^{m \times n}$. Find the orthogonal projection of $A$ onto span($u v^T$).
\end{exercise}

\begin{answer}
\end{answer}

\clearpage

\begin{exercise}
  \textbf{Norm Invariance under Orthogonal Transformations.} Show that for any $A \in \mathbb{R}^{m \times n}$, $Q \in \mathcal{O}_m$, $R \in \mathcal{O}_n$, $\lVert QAR \rVert_F = \lVert A \rVert_F$. Thus the Frovenius norm is invariant under orthogonal transformations. Similarly, show the induced 2-norm of $A \in \mathbb{R}^{m \times n}$ is invariant under orthogonal transformations.
\end{exercise}

\begin{answer}
\end{answer}

\clearpage

\begin{exercise}
  Let $A, B$ be matrices of appropriate size and $x \in \mathbb{R}^n$. Prove that

  \begin{enumerate}[label=\alph*)]
    \item $\lVert A x \rVert_2 \leq \lVert A \rVert_2 \lVert x \rVert_2$ ;
    \item $\lVert A B \rVert_2 \leq \lVert A \rVert_2 \lVert B \rVert_2$.
  \end{enumerate}
\end{exercise}

\begin{answer}
\end{answer}

\clearpage

\begin{exercise}
  For $A, B \in \mathbb{R}^{m \times n}$. Show that $\sigma_1 (A + B) \leq \sigma_1 (A) + \sigma_1 (B)$.
\end{exercise}

\begin{answer}
\end{answer}

\clearpage

\begin{exercise}
  \textbf{The Moore-Penrose pseudo-inverse.} The Moore-Penrose pseudo-inverse of a matrix $A \in \mathbb{R}^{m \times n}$ is the unique matrix $A^+ \in \mathbb{R}^{n \times m}$ satisfying the following four properties:

  \begin{enumerate}[label=\alph*)]
    \item $A(A^+ A) = A$
    \item $(A^+ A)A^+ = A^+$
    \item $(A^+ A)^T = A^+ A$
    \item $(A A^+)^T = A A^+$
  \end{enumerate}

  Let $A$ have compact SVD $A = U \Sigma V^T$. Show that $A^+ = V \Sigma^{-1} U^T$. Give an interpretation of $A^+$ in terms of $\mathcal{N}(A)$, $\mathcal{N}(A)^\perp$, and $\mathcal{R}(A)$.
\end{exercise}

\begin{answer}
\end{answer}

\end{document}