\documentclass[12pt]{article}

\usepackage[utf8]{inputenc}
\usepackage{datetime}
\usepackage{amsthm}
\usepackage{amsmath}
\usepackage{amssymb}
\usepackage{enumitem}
\usepackage[USenglish]{babel}
\usepackage{matlab-prettifier}
\usepackage{graphicx}
\usepackage[makeroom]{cancel}
\usepackage{afterpage}
\usepackage{capt-of}

\DeclareMathOperator*{\argmin}{arg\,min}

\newcommand\independent{\protect\mathpalette{\protect\independenT}{\perp}}
\def\independenT#1#2{\mathrel{\rlap{$#1#2$}\mkern2mu{#1#2}}}

\newtheoremstyle{colon}{\topsep}{\topsep}{}{}{\bfseries}{:}{ }{}
\theoremstyle{colon}
\newtheorem{exercise}{Exercise}
\newtheorem*{answer}{Answer}

\title{ORFE 527: Stochastic Calculus \\ Homework 4}
\author{Zachary Hervieux-Moore}

\newdate{date}{19}{03}{2017}
\date{\displaydate{date}}

\begin{document}

\maketitle

\clearpage

\begin{exercise}
  (Weak solutions) Suppose that the function $\sigma: [0, \infty) \times \mathbb{R} \rightarrow \mathbb{R}$ is such that the SDE
  \begin{gather*}
    \text{d} X_t = \sigma(t, X_t) \text{d} B_t
  \end{gather*}
  has a unique weak solution for any $X_0 \in \mathbb{R}$ (e.g. $\sigma$ is Lipschitz in the $x$ variable). Suppose further that there exists  deterministic constant $c > 0$ such that $\sigma(t,x) \geq c$ for all $(t,x) \in [0,\infty) \times \mathbb{R}$. Show that for any bounded measurable function $b : [0, \infty) \times \mathbb{R} \rightarrow \mathbb{R}$
  \begin{gather*}
    \text{d} X_t = b(t, X_t) \text{d}t + \sigma(t, X_t) \text{d} B_t
  \end{gather*}
  has a unique weak solution for any $X_0 \in \mathbb{R}$.
\end{exercise}

\begin{answer}
  We replicate the steps done in class. Assume there is a standard Brownian motion $X$ on $(\Omega, \mathcal{F}, \mathbb{P})$. Now, define $(\mathcal{F}_t)_{t \geq 0}$ as the filtration generated by $X$. Now define $\mathbb{Q}$ on $(\Omega, \mathcal{F}_T)$
  \begin{gather*}
    \frac{\text{d} \mathbb{Q}}{\text{d} \mathbb{P}} = e^{\int_0^T b(s,X_s)/\sigma(s, X_s) \text{d} X_s - \frac{1}{2} \int_0^T b(s,X_s)^2/\sigma(s, X_s)^2 \text{d} s}
  \end{gather*}
  Since $b$ is bounded and $\sigma \geq c$ then we also have that $b(s,X_s)/\sigma(s, X_s)$ is bounded and so $\mathbb{Q}$ is a probability measure by Novikov's theorem. Thus, by Girsanov's theorem, we have that $B_t := X_t - \int_0^T b(s,X_s)/\sigma(s, X_s) \text{d} s$ is a standard Brownian motion under $\mathbb{Q}$. Thus,
  \begin{gather*}
    \text{d} X_t = b(t,X_t)/\sigma(t, X_t) \text{d} t + \text{d} B_t
  \end{gather*}
  Now, we can use the theorem from class that
  \begin{gather*}
    \text{d} X_t = b(t,X_t) \text{d} t + \text{d} B_t
  \end{gather*}
  has a unique weak solution if $b(t,X_t)$ is bounded. Since, $b(t,X_t)/\sigma(t, X_t) \leq b(t,X_t)/c$ and $b(t,X_t)$ is bounded, we conclude that the original SDE has a unique weak solution.
\end{answer}

\clearpage

\begin{exercise}
  (Variation on Yamada-Watanabe Theorem) Suppose that strong existence holds for the initial value problem
  \begin{gather*}
    \text{d} X_t = b(t, X_t) \text{d} t + \sigma(t, X_t)\text{d} B_t, \ X_0 = x_0
  \end{gather*}
  Suppose further that the joint law of $(X,B)$ is uniquely determined for any solution of this initial value problem. Show that any two strong solutions $X, \tilde{X}$ of the initial value problem with respect to the same Brownian motion $B$ are indistinguishable.
\end{exercise}

\begin{answer}
  By definition of strong solutions,
  \begin{gather*}
    \mathbb{P}(X_t = x_0 + \int_0^t b(s, X_s) \text{d} s + \int_0^t \sigma(s, X_s)\text{d} B_s) = 1 \ \forall t \geq 0 \\
    \mathbb{P}(\tilde{X}_t = x_0 + \int_0^t b(s, \tilde{X}_s) \text{d} s + \int_0^t \sigma(s, \tilde{X}_s)\text{d} B_s) = 1 \ \forall t \geq 0
  \end{gather*}
  That is, $X_t, \tilde{X}_t \in C([0,\infty), \mathbb{R})$. Also, by uniqueness of the joint law, we have for $A \in C([0,\infty), \mathbb{R})$
  \begin{gather*}
    \mathbb{E}[X_t 1_{B_{[0,t]} \in \{A\}}] = \mathbb{E}[\tilde{X}_t 1_{B_{[0,t]} \in \{A\}}] \\
    \Longleftrightarrow \mathbb{E}[(X_t - \tilde{X}_t) 1_{B_{[0,t]} \in \{A\}}] = 0 \ \forall A
  \end{gather*}
  Therefore $X_t = \tilde{X}_t$ a.s. and we have
  \begin{gather*}
    \mathbb{P}(X_t = \tilde{X}_t \ \forall t \in \mathbb{Q})
  \end{gather*}
  Since the paths are continuous, we can expand $t$ to all real numbers
  \begin{gather*}
    \mathbb{P}(X_t = \tilde{X}_t \ \forall t \in \mathbb{R})
  \end{gather*}
  Which implies that $X_t$ and $\tilde{X}_t$ are indistinguishable.
\end{answer}

\clearpage

\begin{exercise}
  (Bougerol's identity) Let $B^{(1)}, B^{(2)}$ be independent standard Brownian motions.
  \begin{enumerate}[label=\alph*)]
    \item Find the SDE satisfied by the process
      \begin{gather*}
        X_t := e^{B_t^{(1)}} \int_0^t e^{-B_s^{(1)}} \text{d} B_s^{(2)}, \ t \geq 0
      \end{gather*}

    \item Find the SDE satisfied by the process $Y_t := \sinh B_t^{(1)}$, $t \geq 0$

    \item Use (a) and (b) to show the identity in distribution
      \begin{gather*}
        \int_0^t e^{B_s^{(1)}} \text{d} B_s^{(2)} \stackrel{d}{=} \sinh B_t^{(1)}
      \end{gather*}
      for any fixed $t \geq 0$. The latter is known as \textit{Bougerol's identity}.
  \end{enumerate}
\end{exercise}

\begin{answer}
  \leavevmode
  \begin{enumerate}[label=\alph*)]
    \item We apply the multivariate Ito's formula to $X_t$
      \begin{gather*}
        \text{d} X_t = \left( e^{B_t^{(1)}} \int_0^t e^{-B_s^{(1)}} \text{d} B_s^{(2)} \right) \text{d} B_t^{(1)} + \left( e^{B_t^{(1)}} e^{-B_t^{(1)}} \right) \text{d} B_t^{(2)} \\
        + \left( e^{B_t^{(1)}} \int_0^t e^{-B_s^{(1)}} \text{d} B_s^{(2)} \right) \text{d} t \\
        = X_t \text{d} B_t^{(1)} + \text{d} B_t^{(2)} + X_t \text{d} t
      \end{gather*}

    \item Apply Ito's formula to $Y_t$ and use hyperbolic trig identity
      \begin{gather*}
        \text{d} Y_t = \cosh \left( B_t^{(1)} \right) \text{d} B_t^{(1)} + \sinh \left( B_t^{(1)} \right) \text{d} t \\
        = \sqrt{1 + \sinh^2 \left( B_t^{(1)} \right)} \text{d} B_t^{(1)} + Y_t \text{d} t \\
        = \sqrt{1 + Y_t^2} \text{d} B_t^{(1)} + Y_t \text{d} t
      \end{gather*}

    \item We define
      \begin{gather*}
        \text{d} Z_t = \frac{\text{d} B_t^{(2)} + X_t \text{d} B_t^{(1)}}{\sqrt{1 + X_t^2}}
      \end{gather*}
      Note that this is a local martingale. We compute its quadratic variation.
      \begin{gather*}
        \text{d} \langle Z \rangle_t = \frac{\text{d} t}{1 + X_t^2} + \frac{X_t^2 \text{d} t}{1 + X_t^2} \\
        = \frac{ (1 + X_t^2) \text{d} t}{1 + X_t^2} \\
        = \text{d} t
      \end{gather*}
      Thus, by Levy's characterization, $Z_t$ is a Brownian motion. Furthermore, using part (a),
      \begin{gather*}
        \text{d} X_t = X_t \text{d} B_t^{(1)} + \text{d} B_t^{(2)} + X_t \text{d} t \\
        \implies \text{d} X_t = \sqrt{1 - X_t^2} \text{d} Z_t + X_t \text{d} t
      \end{gather*}
      Thus, $X_t$ and $Y_t$ satisfy the same stochastic differential equation. Thus, $X_t \stackrel{d}{=} Y_t = \sinh B_t^{(1)}$. Now, we show that $X_t$ is the same in distribution as the result we wish.
      \begin{gather*}
        X_t = e^{B_t^{(1)}} \int_0^t e^{-B_s^{(1)}} \text{d} B_s^{(2)} \\
        = \int_0^t e^{B_t^{(1)}-B_s^{(1)}} \text{d} B_s^{(2)} \\
        = \int_0^t e^{B_{t-s}^{(1)}} \text{d} B_s^{(2)}
      \end{gather*}
      Change of variable $s' = t - s$
      \begin{gather*}
        = - \int_t^0 e^{B_{s'}^{(1)}} \text{d} B_{s'}^{(2)} \\
        = \int_0^t e^{B_{s}^{(1)}} \text{d} B_{s}^{(2)}
      \end{gather*}
      Thus, we have shown precisely that $\int_0^t e^{B_s^{(1)}} \text{d} B_s^{(2)} = X_t \stackrel{d}{=} Y_t = \sinh B_t^{(1)}$
  \end{enumerate}
\end{answer}

\end{document}